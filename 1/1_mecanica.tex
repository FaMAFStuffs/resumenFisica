\section{Física y medición}
  \subsection{Estándares de longitud, masa y tiempo}
    \PN En mecánica, las tres cantidades fundamentales son longitud, masa y tiempo. Todas las otras cantidades en
    mecánica se pueden expresar en función de estas tres.

    \PN En 1960, un comité internacional estableció un conjunto de estándares para las cantidades fundamentales de la
    ciencia. Se llama \textbf{Sistema Internacional} (SI)) y sus unidades fundamentales de \textbf{longitud},
    \textbf{masa} y \textbf{tiempo} son metro, kilogramo y segundo, respectivamente. Otros estándares para las unidades
    fundamentales SI establecidas por el comité son las de temperatura (el \textit{kelvin}), corriente eléctrica (el
    \textit{ampere}), etc.
    \begin{itemize}
      \item Longitud: La distancia entre dos puntos en el espacio se identifica como \textbf{longitud}. El
      \textbf{metro} se redefinió como la distancia recorrida por la luz en el vacío durante un tiempo de 1/299 792 458
      segundos.
      \item Masa: La unidad fundamental de masa en el SI, el \textbf{kilogramo} (kg), está definida como la masa de un
      cilindro de aleación platino–iridio específico.
      \item Tiempo: La unidad fundamental es el \textbf{segundo} que se define como 9 192 631 770 veces el periodo de
      vibración de la radiación del átomo de cesio 133.
    \end{itemize}
