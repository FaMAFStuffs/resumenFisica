\section{Física y medición}
  \subsection{Estándares de longitud, masa y tiempo}
    \PN En mecánica, las tres cantidades fundamentales son longitud, masa y tiempo. Todas las otras cantidades en
    mecánica se pueden expresar en función de estas tres.

    \PN En 1960, un comité internacional estableció un conjunto de estándares para las cantidades fundamentales de la
    ciencia. Se llama \textbf{Sistema Internacional} (SI)) y sus unidades fundamentales de \textbf{longitud},
    \textbf{masa} y \textbf{tiempo} son \textit{metro, kilogramo y segundo}, respectivamente. Otros estándares para las
    unidades fundamentales SI establecidas por el comité son las de temperatura (el \textit{kelvin}), corriente
    eléctrica (el \textit{ampere}), etc.
    \begin{itemize}
      \item \underline{Longitud:} La distancia entre dos puntos en el espacio se identifica como \textbf{longitud}. El
      \textbf{metro} se redefinió como la distancia recorrida por la luz en el vacío durante un tiempo de 1/299 792 458
      segundos.
      \item \underline{Masa:} La unidad fundamental de masa en el SI, el \textbf{kilogramo} (kg), está definida como la
      masa de un cilindro de aleación platino–iridio específico.
      \item \underline{Tiempo:} La unidad fundamental es el \textbf{segundo} que se define como 9 192 631 770 veces el
      periodo\footnote{Intervalo de tiempo necesario para una vibración completa} de vibración de la radiación del átomo
      de cesio 133.
    \end{itemize}

  \subsection{Cifras significativas}
    \begin{itemize}
      \item Cuando se multiplican muchas cantidades, el número de cifras significativas en la respuesta final es el
      mismo que el número de cifras significativas en la cantidad que tiene el número más pequeño de cifras
      significativas. La misma regla aplica para la división.
      \item Cuando los números se sumen o resten, el número de lugares decimales en el resultado debe ser igual al
      número más pequeño de lugares decimales de cualquier término en la suma o resta.
    \end{itemize}
