\section{Vectores}
  \subsection{Sistemas coordenados}
    \PN En el sistema de coordenadas polares $(r, \theta)$, $r$ es la distancia desde el origen hasta el punto que tiene
    coordenadas cartesianas (x, y) y $\theta$ es el ángulo entre un eje fijo y una recta dibujada desde el origen hasta
    el punto. El eje fijo es el eje x positivo y $\theta$ se mide contra el sentido de las agujas del reloj.
    \begin{eqnarray*}
      x = r \ cos(\theta) \\
      y = r \ sen(\theta) \\
      r = \sqrt{x^{2} + y^{2}}
    \end{eqnarray*}

  \subsection{Cantidades vectoriales y escalares}
    \begin{itemize}
      \item Una \textbf{cantidad escalar} se especifica por completo mediante un valor único con una unidad adecuada y
      no tiene dirección.
      \item Una \textbf{cantidad vectorial} se especifica por completo mediante un número y unidades apropiadas más una
      dirección.
    \end{itemize}
