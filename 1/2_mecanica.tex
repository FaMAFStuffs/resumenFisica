\section{Movimiento en una dimensión}
  \subsection{Posición, velocidad y repidez}
    \begin{itemize}
      \item La \textbf{posición} x de una partícula es la ubicación de la partícula respecto a un punto de referencia
      elegido que se considera el origen de un sistema coordenado. El movimiento de una partícula se conoce por completo
      si la posición de la partícula en el espacio se conoce en todo momento.

      \item El \textbf{desplazamiento} $\Delta x$ de una partícula se define como su cambio en posición en algún
      intervalo de tiempo. Conforme la partícula se mueve desde una posición inicial $x_{i}$ a una posición final
      $x_{f}$, su desplazamiento está dado por:
      \begin{equation*}
        \Delta x \equiv x_{f} - x_{i}
      \end{equation*}

      \item \textbf{Distancia} es la longitud de una trayectoria seguida por una partícula. La distancia siempre se
      representa como un número positivo, mientras que el desplazamiento puede ser positivo o negativo.

      \item La \textbf{velocidad promedio} $v_{x,prom}$ de una partícula se define como el desplazamiento x de la
      partícula dividido entre el intervalo de tiempo $t$ durante el que ocurre dicho desplazamiento:
      \begin{equation*}
        v_{x,prom} \equiv \frac{\Delta x}{\Delta t}
      \end{equation*}

      \item La \textbf{rapidez promedio} $v_{prom}$ de una partícula, una cantidad escalar, se define como la distancia
      total recorrida dividida entre el intervalo de tiempo total requerido para recorrer dicha distancia:
      \begin{equation*}
        v_{prom} \equiv \frac{d}{\Delta t}
      \end{equation*}
    \end{itemize}

  \subsection{Velocidad y rapidez instantáneas}
    \begin{itemize}
      \item La \textbf{velocidad instantánea} $v_{x}$ es igual al valor límite de la razón $\Delta x / \Delta t$
      conforme $\Delta t$ tiende a cero, es decir, a la derivada de x respecto de $t$:
      \begin{equation*}
        v_{x} \equiv \lim_{\Delta t \rightarrow 0} \ \frac{\Delta x}{\Delta t} = \frac{dx}{dt}
      \end{equation*}

      \item La \textbf{rapidez instantánea} de una partícula se define como la magnitud de su velocidad instantánea.
    \end{itemize}

  \subsection{Aceleración}
    \begin{itemize}
      \item La \textbf{aceleración promedio} $a_{x,prom}$ de la partícula se define como el cambio en velocidad $\Delta
      v_{x}$ dividido entre el intervalo de tiempo $t$ durante el que ocurre el cambio:
      \begin{equation*}
        a_{x,prom} \equiv \frac{\Delta v_{x}}{\Delta t}
      \end{equation*}

      \item La \textbf{aceleción instantánea} como el límite de la aceleración promedio conforme $\Delta t$ tiende a
      cero.
      \begin{equation*}
        a_{x} \equiv \lim_{\Delta t \rightarrow 0} \ \frac{\Delta v_{x}}{\Delta t} = \frac{dv_{x}}{dt}
      \end{equation*}
    \end{itemize}

  \subsection{Análisis de modelo: la partícula bajo aceleración constante}
    \begin{itemize}
      \item \textbf{Posición:}
      \begin{equation*}
        x_{f} = x_{i} + v_{xi} t + \frac{1}{2} a_{x}t^{2}
      \end{equation*}
      \item \textbf{Velocidad:}
      \begin{equation*}
        v_{xf} = v_{xi} + a_{x} t
      \end{equation*}
    \end{itemize}
