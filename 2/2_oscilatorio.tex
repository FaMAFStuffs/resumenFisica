\subsection{Partícula en movimiento armónico simple}

  \PN La posición de un objeto actuando sobre una fuerza descrita por la \textit{Ley de Hooke} esta dada por:
  \begin{equation}
    x(t) = A \cos (\omega t)
  \end{equation}

  \PN donde A, $\omega$, son constantes y:
  \begin{itemize}
    \item Amplitud del movimiento (A): es simplemente el máximo valor de la posición de la partícula en la dirección x
    positiva o negativa.
    \item Frecuencia angular ($\omega$): es una medida de qué tan rápido se presentan las oscilaciones;
    mientras más oscilaciones por unidad de tiempo haya, más alto es el valor.
    \begin{equation}
      \omega = \sqrt{\frac{k}{m}}
    \end{equation}
    \item Período del movimiento (T): es el intervalo de tiempo requerido para que la partícula pase a través de un
    ciclo completo de su movimiento.
    \begin{equation}
      T = \frac{2\pi}{\omega} = 2\pi \sqrt{\frac{m}{k}}
    \end{equation}
    \item Frecuencia (f): representa el número de oscilaciones que experimenta la partícula por unidad de intervalo de
    tiempo.
    \begin{equation}
      f = \frac{1}{T} = \frac{1}{2\pi} \sqrt{\frac{k}{m}}
    \end{equation}
  \end{itemize}

  \PN \textbf{Ecuaciones de velocidad y de aceleración}
  \begin{eqnarray*}
    v = \frac{dx}{dt} &=& - \omega A \sin (\omega t) \\
    a = \frac{d^{2}x}{dt^{2}} &=& - \omega^{2} A \cos (\omega t)
  \end{eqnarray*}

  \PN \textbf{Valores máximos}
  \begin{eqnarray*}
    x_{max} &=& A \\
    v_{max} &=& \omega A = \sqrt{\frac{k}{m}} \\
    a_{max} &=& \omega^{2} A = \frac{k}{m} A
  \end{eqnarray*}
