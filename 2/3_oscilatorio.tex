\subsection{Energía del oscilador armónico simple}

\PN \textbf{Energía cinética}
\begin{equation}
  K = \frac{1}{2} m v^{2} = \frac{1}{2} m \omega^{2} A^{2} \sin^{2} (\omega t)
\end{equation}

\PN \textbf{Energía potencial}
\begin{equation}
  U = \frac{1}{2} k x^{2} = \frac{1}{2} k A^{2} \cos^{2} (\omega t)
\end{equation}

\PN \textbf{Energía total}
\PN Dado que K y U siempre son cantidades positivas o cero. Puesto que $\omega^{2} = \frac{k}{m}$, la energía mecánica
total del oscilador armónico simple se expresa como:
\begin{eqnarray*}
  E &=& K + U = \frac{1}{2} k A^{2} [\sin^{2} (\omega t) + \cos^{2} (\omega t)] \\
  &=& \frac{1}{2} k A^{2}
\end{eqnarray*}

\pagebreak
\PN \textbf{Velocidad como una función de la posición}
\PN La velocidad del bloque en una posición arbitraria se obtiene al expresar la energía total del sistema en alguna
posición arbitraria x como:
\begin{eqnarray*}
  E &=& K + U = \frac{1}{2} m v^{2} + \frac{1}{2} k x^{2} = \frac{1}{2} k A^{2} \\
  v &=& \pm \sqrt{\frac{k}{m} (A^{2} - x^{2})} = \pm \omega \sqrt{A^{2} - x^{2}}
\end{eqnarray*}
