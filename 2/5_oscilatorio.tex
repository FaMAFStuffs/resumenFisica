\subsection{Oscilaciones forzadas}

  \PN Se ha visto que la energía mecánica de un oscilador amortiguado disminuye en el tiempo como resultado de la fuerza
  retardadora. Es posible compensar esta disminución de energía al aplicar una fuerza externa que haga trabajo positivo
  sobre el  sistema. En cualquier instante se puede transferir energía al sistema mediante una fuerza aplicada que actúe
  en la dirección de movimiento del oscilador.

  \PN Al modelar un oscilador con fuerzas retardadoras e impulsoras como una partícula bajo una fuerza neta, la segunda
  ley de Newton en esta situación produce:
  \begin{equation}
    \Sigma \ F_{x} = ma_{x} \rightarrow F_{0} \sin (\omega t) -b \frac{dx}{dt} - kx = m \frac{d^{2}x}{dt^{2}}
  \end{equation}

  \PN \textbf{Ecuación de la posición}
  \begin{equation}
    x(t) = A \cos (\omega t)
  \end{equation}

  \PN donde
  \begin{equation}
    A = \frac{F_{0}/m}{\sqrt{\left(\omega^{2} - \omega_{0}^{2}\right)^{2} - \left(\frac{b\omega}{m}\right)^{2}}}
  \end{equation}

  \PN y donde $\omega_{0} = \sqrt{k/m}$ es la frecuencia natural del oscilador subamortiguado ($b = 0$).
