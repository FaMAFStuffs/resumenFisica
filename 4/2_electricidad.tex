\section{Ley de Gauss}
  \subsection{Flujo eléctrico}
    \PN El total de líneas que penetran en la superficie es proporcional al producto $EA$. A este producto de la
    magnitud del campo eléctrico $E$ y al área superficial $A$, perpendicular al campo, se le conoce como \textbf{flujo
    eléctrico} $\Phi$.
    \begin{equation*}
      \Phi = EA \cos(\theta)
    \end{equation*}

    \PN Con base en las unidades del SI correspondientes a $E$ y $A$, $\Phi$ se expresa en (N $m^{2} /$ C). El flujo
    eléctrico es proporcional al número de las líneas de campo eléctrico que penetran en una superficie.

    \PN Considere una superficie dividida en un gran número de elementos pequeños, cada uno de área $\Delta A$. Es
    conveniente definir un vector $\Delta \vec{A}_{i}$ cuya magnitud representa el área del elemento i-ésimo sobre la
    superficie y cuya dirección está definida como perpendicular al elemento de superficie. El campo eléctrico $\vec{E}$
    en la ubicación de este elemento forma un ángulo $u_{i}$ con el vector $\Delta \vec{A}_{i}$. El flujo eléctrico
    $\Delta \Phi$ a través de este elemento es
    \begin{equation*}
      \Delta \Phi = E_{i} \Delta A_{i} \cos(\theta_{i}) = \vec{E} \Delta \vec{A}_{i}
    \end{equation*}

    \PN Al sumar las contribuciones de todos los elementos, obtiene el flujo total a través de la superficie.
    \begin{equation*}
      \Phi \approx \sum \vec{E}_{i} \Delta \vec{A}_{i}
    \end{equation*}

    \PN Si supone que el área de cada elemento se acerca a cero, en tal caso el número de elementos se acercaría al
    infinito y la suma se reemplaza por una integral. Debido a eso, la definición general del flujo eléctrico es
    \begin{equation*}
      \Phi = \oint \vec{E} \ d\vec{A}
    \end{equation*}

    \PN El flujo neto a través de la superficie es proporcional al número neto de líneas que salen de la superficie,
    donde número neto significa la cantidad de líneas que salen de la superficie menos la cantidad de líneas que entran.

    \subsection{Ley de Gauss}
      \PN La Ley de Gauss se define como la correspondencia de tipo general entre el flujo eléctrico neto a través de
      una superficie cerrada y la carga encerrada en la superficie.

      \PN Suponga de nuevo una carga puntual positiva $q$ ubicada en el centro de una esfera de radio $r$. Se sabe que
      la magnitud del campo eléctrico en todos los puntos de la superficie de la esfera es $E = k q/r^{2}$. Las líneas
      de campo están dirigidas radialmente hacia afuera y por tanto son perpendiculares a la superficie en todos sus
      puntos. Es decir, en cada punto de la superficie, $\vec{E}$ es paralelo al vector $\Delta \vec{A}$ que representa
      un elemento de área local $\Delta A_{i}$ que rodea al punto en la superficie. Por lo tanto
      \begin{equation*}
        \vec{E} \ \Delta \vec{A}_{i} = E \ \Delta A
      \end{equation*}

      \PN y el flujo neto a través de la superficie gaussiana es igual a
      \begin{equation*}
        \Phi = \oint \vec{E} d\vec{A} = \oint E dA = E \oint dA
      \end{equation*}

      \PN donde se ha retirado $E$ afuera de la integral ya que, por simetría, $E$ es constante en la superficie y se
      conoce por $E = k q/r^{2}$. Además, en vista de que la superficie es esférica, $\oint dA = A = 4 \pi r^{2}$. Por
      lo tanto, el flujo neto a través de la superficie gaussiana es
      \begin{equation*}
        \Phi = k \frac{q}{r^{2}} (4 \pi r^{2}) = 4 \pi kq
      \end{equation*}

      \PN Dado que $k = 1/4 \pi \varepsilon_{0}$, escribimos
      \begin{equation*}
        \Phi = \frac{q}{\varepsilon_{0}}
      \end{equation*}

      \PN \textbf{\underline{Observaciones}}
      \begin{itemize}
        \item el flujo neto a través de cualquier superficie cerrada que rodea a una carga puntual $q$ tiene un valor de
        $q/e_{0}$ y es independiente de la forma de la superficie.
        \item el flujo eléctrico neto a través de una superficie cerrada que no rodea a ninguna carga es igual a cero.
      \end{itemize}

    \subsection{Conductores en equilibrio estático}
      \PN Un buen conductor eléctrico contiene cargas (electrones) que no se encuentran unidas a ningún átomo y debido a
      eso tienen la libertad de moverse en el interior del material. Cuando dentro de un conductor no existe ningún
      movimiento neto de carga, el conductor está en \textbf{equilibrio electrostático}. Un conductor en equilibrio
      electrostático tiene las siguientes propiedades:
      \begin{itemize}
        \item En el interior del conductor el campo eléctrico es cero, si el conductor es sólido o hueco.
        \item Si un conductor aislado tiene carga, ésta reside en su superficie.
        \item El campo eléctrico justo fuera de un conductor con carga es perpendicular a la superficie del conductor y
        tiene una magnitud $s/e_{0}$, donde $s$ es la densidad de carga superficial en ese punto.
        \item En un conductor de forma irregular, la densidad de carga superficial es máxima en aquellos puntos donde el
        radio de curvatura de la superficie es el menor.
      \end{itemize}
