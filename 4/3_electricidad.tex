\section{Potencial eléctrico}
  \PN Cuando se coloca una carga de prueba $q_{0}$ en un campo eléctrico $\vec{E}$ producido por alguna distribución de
  carga fuente, la fuerza eléctrica que actúa sobre ella es $q_{0} \vec{E}$. La fuerza $q_{0}\vec{E}$ es conservativa,
  ya que la fuerza entre cargas descrita por la ley de Coulomb es conservativa. Cuando se traslada la carga de prueba
  por algún agente externo en el campo, el trabajo consumido por el campo en la carga es igual al trabajo invertido por
  el agente externo que origina el desplazamiento, pero con signo negativo.

  \VS
  \PN Al analizar los campos eléctricos y magnéticos, es común utilizar la notación $d\vec{s}$ para representar un
  vector de desplazamiento infinitesimal que tiene una orientación tangente a una trayectoria a través del espacio. Esta
  trayectoria puede ser recta o curva, y la integral calculada a lo largo de esta trayectoria se conoce como integral de
  la trayectoria, o bien, integral de línea.

  \VS
  \PN Para un desplazamiento infinitesimal $d\vec{s}$ de una carga puntual $q_{0}$ inmersa en un campo eléctrico, el
  trabajo realizado por un campo eléctrico sobre la misma es $\vec{F} d\vec{s} = q_{0} \vec{E} ds$. Conforme el campo
  consume esta cantidad de trabajo, la energía potencial del sistema carga-campo cambia en una cantidad $dU = -q_{0}
  \vec{E} d\vec{s}$. Para un desplazamiento finito de la carga desde el punto A al punto B, el cambio en energía
  potencial del sistema $\Delta U = U_{B} - U_{A}$ es
  \begin{equation*}
    \Delta U = -q_{0} \int_{A}^{B} \vec{E} \ d\vec{s}
  \end{equation*}

  \PN La integración se lleva a cabo a lo largo de la trayectoria que $q_{0}$ sigue al pasar de A a B. Porque la fuerza
  $q_{0}\vec{E}$ es conservativa, la integral de línea no depende de la trayectoria de A a B.

  \VS
  \PN Al dividir la energía potencial entre la carga de prueba se obtiene una cantidad física que depende sólo de la
  distribución de carga fuente y tiene un valor en cada uno de los puntos de un campo eléctrico. Esta cantidad se conoce
  como \textbf{potencial eléctrico} ($V$):
  \begin{equation*}
    V = \frac{U}{q_{0}}
  \end{equation*}

  \PN Si la carga de prueba es desplazada entre las posiciones A y B en un campo eléctrico, el sistema carga-campo
  experimenta un cambio en su energía potencial. La \textbf{diferencia de potencial} $\Delta V = V_{B} - V_{A}$ entre
  los puntos A y B de un campo eléctrico se define como el cambio en energía potencial en el sistema al mover una carga
  de prueba $q_{0}$ entre los puntos, dividido entre la carga de prueba:
  \begin{equation*}
    \Delta V \equiv \frac{\Delta U}{q_{0}} = - \int_{A}^{B} \vec{E} \ d\vec{s}
  \end{equation*}

  \PN Si un agente externo traslada una carga de prueba de A a B sin modificar la energía cinética de ésta, el agente
  realiza un trabajo que modifica la energía potencial del sistema: $W = \Delta U$. Imagine una carga $q$ arbitraria
  localizada en un campo eléctrico. El trabajo consumido por un agente externo al desplazar una carga $q$ a través de un
  campo eléctrico con una velocidad constante es
  \begin{equation*}
    W = q \Delta V
  \end{equation*}

  \PN Ya que el potencial eléctrico es una medida de la energía potencial por unidad de carga, la unidad del SI, tanto
  del potencial eléctrico como de la diferencia de potencial, es joules por cada coulomb, que se define como un
  \textbf{volt} (V):
  \begin{equation*}
    1 \ \text{V} \equiv 1 \ \frac{\text{J}}{\text{C}}
  \end{equation*}

  \PN Como conclusión, el campo eléctrico es una medida de la relación de cambio en función de la posición del potencial
  eléctrico.
