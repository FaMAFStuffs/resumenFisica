\subsection{Descripción macroscópica de un gas ideal}

  \PN La ecuación de expansión volumétrica $\Delta V = \beta V_{i} \Delta T$ se basa en la suposición de que el material
  tiene un volumen inicial $V_{i}$ antes de que ocurra un cambio de temperatura. Tal es el caso para sólidos y líquidos,
  porque tienen un volumen fijo a una temperatura dada.

  \PN Para un gas, es útil saber cómo se relacionan las cantidades volumen $V$, presión $P$ y temperatura $T$ para una
  muestra de gas de masa $m$. En general, la ecuación que interrelaciona estas cantidades, llamada \textit{ecuación de
  estado}, es muy complicada. Sin embargo, si el gas se mantiene a una presión muy baja (o densidad baja), la ecuación
  de estado es muy simple y se puede determinar a partir de resultados experimentales. Tal gas de densidad baja se
  refiere como un gas ideal. El modelo de gas ideal se puede emplear para efectuar predicciones que sean adecuadas para
  describir el comportamiento de gases reales a bajas presiones.

  \vspace{3mm}
  \PN \textbf{Ley de gas ideal}
  \begin{equation*}
    P V = c T
  \end{equation*}

  \PN donde c, es una constante que se define como $c = nR$, con $n$ el número de \textit{moles} de la sustancia y $R$
  se llama \textbf{constante universal de los gases} y tiene el valor:
  \begin{equation*}
    R = 0.082 \; \frac{L \; atm}{mol \; K}
  \end{equation*}
