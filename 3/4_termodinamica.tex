\subsection{Expansión térmica de sólidos y líquidos}

  \PN El estudio del termómetro líquido utiliza uno de los cambios mejor conocidos en una sustancia: a medida que
  aumenta su temperatura, su volumen se incrementa. Este fenómeno, conocido como \textbf{expansión térmica}, desempeña
  un importante papel en numerosas aplicaciones de ingeniería. La expansión térmica es una consecuencia del cambio en la
  separación promedio entre los átomos en un objeto.

  \PN Suponga que un objeto tiene una longitud inicial $L_{i}$ a lo largo de alguna dirección en alguna temperatura, y
  la longitud aumenta en una cantidad $\Delta L$ para un cambio en temperatura $\Delta T$. Es conveniente considerar el
  cambio fraccionario en longitud por cada grado de cambio de temperatura, entonces se define el \textit{coeficiente de
  expansión lineal} promedio como:
  \begin{equation*}
    \alpha \equiv \frac{\Delta L / L_{i}}{\Delta T}
  \end{equation*}

  \PN Para fines de cálculo, esta ecuación se reescribe como:
  \begin{equation*}
    \Delta L = \alpha L_{i} \Delta T
  \end{equation*}

  \PN o en la forma:
  \begin{equation*}
    L_{f} - L_{i} = \alpha L_{i} \left(T_{f} - T_{i}\right)
  \end{equation*}

  \PN donde $L_{f}$ es la longitud final, $T_{i}$ y $T_{f}$ son las temperaturas inicial y final, respectivamente, y la
  constante de proporcionalidad a es el coeficiente de expansión lineal promedio para un material dado y tiene unidades
  de $(ºC)^{-1}$.

  \PN Ya que las dimensiones lineales de un objeto cambian con la temperatura, se sigue que el área superficial y el
  volumen cambian. El cambio en volumen es proporcional al volumen inicial $V_{i}$ y al cambio en temperatura de acuerdo
  con la relación:
  \begin{equation*}
    \Delta V = \beta V_{i} \Delta T
  \end{equation*}

  \PN Comunmente, se cumple que $\beta = 3 \alpha$.
