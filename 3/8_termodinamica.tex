\subsection{Calor latente}

  \PN \underline{Convención:} se utilizará el término \textit{material de fase superior} para indicar que el material
  está a una temperatura alta, cuando se estudien dos fases de un material.

  \vspace{3mm}
  \PN Considere un sistema que contiene una sustancia en dos fases en equilibrio, como hielo y agua. La cantidad inicial
  de material de fase superior, agua, en el sistema es $m_{i}$. Ahora imagine que al sistema entra la energía Q. Como
  resultado, la cantidad final de agua es $m_{f}$ debido a la fusión de un poco de hielo. Por lo tanto, la cantidad de
  hielo derretido es igual a la cantidad de agua nueva: $\Delta m = m_{f} - m_{i}$. Para este cambio de fase, el calor
  latente se define como:
  \begin{equation*}
    \text{L} \equiv \frac{\text{Q}}{\Delta m}
  \end{equation*}

  \PN De la definición de calor latente, y de nuevo al elegir el calor como el mecanismo de transferencia de energía, la
  energía requerida para cambiar la fase de una sustancia pura es:
  \begin{equation*}
    \text{Q} = \text{L} \; \Delta m
  \end{equation*}

  \PN donde $\Delta m$ es el cambio en masa del material de fase superior.

  \begin{itemize}
    \item \textbf{Calor latente de fusión} ($L_{f}$): es el término que se aplica cuando el cambio de fase es de sólido
    a líquido.
    \item \textbf{Calor latente de vaporización:} ($L_{v}$): es el término que se usa cuando el cambio de fase es de
    líquido a gas.
  \end{itemize}
