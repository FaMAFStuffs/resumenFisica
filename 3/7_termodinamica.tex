\subsection{Calor específico y calorimetría}

  \PN La \textbf{capacidad térmica} (o calorífica) C de una muestra particular se define como la cantidad de energía
  necesaria para elevar la temperatura de dicha muestra en $1º$C. A partir de esta definición, se ve que si la energía
  $Q$ produce un cambio $\Delta T$ en la temperatura de una muestra, entonces
  \begin{equation*}
    Q = C \Delta T = C (T_{f} - T_{i})
  \end{equation*}

  \PN El \textbf{calor específico} $c$ de una sustancia es la capacidad térmica por unidad de masa. Por lo tanto, si a
  una muestra de una sustancia con masa $m$ se le transfiere energía $Q$ y la temperatura de la muestra cambia en
  $\Delta T$, el calor específico de la sustancia es:
  \begin{equation*}
    c \equiv \frac{Q}{m \Delta T} \qquad \text{i.e} \qquad C = mc
  \end{equation*}

  \PN donde
  \begin{equation*}
    \left[c\right] = \frac{\text{cal}}{\text{g} º\text{C}} = 4.186 \frac{\text{J}}{\text{kg K}}
  \end{equation*}

  \PN El calor específico es en esencia una medida de qué tan insensible térmicamente es una sustancia a la adición de
  energía.

  \vspace{3mm}
  \PN A partir de esta definición, es factible relacionar la energía $Q$ transferida entre una muestra de masa $m$ de un
  material y sus alrededores con un cambio de temperatura $\Delta T$ como:
  \begin{equation*}
    Q = mc \Delta T
  \end{equation*}
