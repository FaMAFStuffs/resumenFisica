\subsection{Calor y energía interna}
  \PN La \textbf{energía interna} es toda la energía de un sistema que se asocia con sus componentes microscópicos,
  átomos y moléculas, cuando se observa desde un marco de referencia en reposo respecto al centro de masa del sistema.

  \vspace{3mm}
  \PN El \textbf{calor} se define como un proceso de transferencia de energía a través de la frontera de un sistema
  debido a una diferencia de temperatura entre el sistema y sus alrededores. También es la cantidad de energía Q
  transferida mediante este proceso.

  \vspace{3mm}
  \PN La \textbf{caloría} (cal), que se define como la cantidad de transferencia de energía necesaria para elevar legal
  temperatura de 1 $g$ de agua de $14.5º$C a $15.5º$C.

  \vspace{3mm}
  \PN \textbf{Equivalente mécanico del calor}
  \begin{equation*}
    1 \; \text{cal} = 4.186 \; \text{J}
  \end{equation*}

  \PN Un nombre más conveniente sería \textit{equivalencia entre energía mecánica y energía interna}, pero el nombre
  histórico tiene mucha presencia en el lenguaje cotidiano, a pesar del uso incorrecto de la palabra calor.
