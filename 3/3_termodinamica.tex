\subsection{Escala absoluta de temperatura}

  \PN \underline{Cero absoluto:} se establece al valor $-273.15º$C para la \textbf{escala absoluta de temperatura}.

  \vspace{3mm}
  \PN El tamaño de un grado en la escala absoluta de temperatura se elige como idéntica al tamaño de un grado en la
  escala Celsius. Por lo tanto, la conversión entre dichas temperaturas es:
  \begin{equation*}
    T_{C} = T - 273.15
  \end{equation*}

  \PN donde $T_{C}$ es la temperatura Celsius y T es la temperatura absoluta.

  \vspace{3mm}
  \PN \textbf{Las escalas de temperatura Celsius, Fahrenheit y Kelvin}
  \PN La \textit{escala Fahrenheit} ubica la temperatura del punto de hielo en $32º$F y la temperatura del punto de
  vapor en $212º$F. La relación entre las escalas de temperatura Celsius y Fahrenheit es:
  \begin{equation*}
    T_{F} = \frac{9}{5} T_{C} + 32º F
  \end{equation*}

  \PN La relación entre los cambios de temperatura en las escalas Celsius, Kelvin y Fahrenheit:
  \begin{equation*}
    \Delta T_{C} = \Delta T = \frac{5}{9} \Delta T_{F}
  \end{equation*}

  \PN De estas tres escalas de temperatura, sólo la escala Kelvin se apoya en un verdadero valor cero de temperatura.
